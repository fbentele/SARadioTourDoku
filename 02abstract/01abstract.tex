\chapter*{Abstract}

Bisher mussten bei Radrennen Motorradfahrer die Nummern der Fahrer, welche aus dem Feld ausgerissen sind, dem \textit{RadioTour Speaker} melden. Dieser meldet dann die Ausreisser jeweils per Funk an die Mannschaftsleiter weiter. Bei den meisten Radrennen notieren die \textit{RadioTour Speaker} die erhaltenen Informationen in ein Notizbuch, bevor sie die Fahrernummern weitergeben. An der Tour de Suisse wir seit einigen Jahren mit einer Tablet-PC Web Applikation zur elektronischen Erfassung der Renninformationen gearbeitet.
\\
Die im Rahmen dieser Studienarbeit realisierte \textit{RadioTour} Android Tablet Applikation ersetzt und erweitert die "`in die Jahre gekommene"' webbasierte Tablet-PC Anwendung. Die neue native \textit{RadioTour} Android Anwendung bietet eine einfachere Touchscreen Bedienung, verbesserte Importfunktionen sowie neue Features wie beispielsweise Live-Marschtabellen und Streckenkilometerbestimmung aus den lokalen GPS-Daten. Verbessert wurde auch die Kommunikation mit der TourLive Webseite via \gls{json} zur unmittelbaren Veröffentlichung der aktuellen Rennsituationen. Die Anwendung ist bewusst für Android Honeycomb \& Ice Cream Sandwich und für die  spezifische Hardware-Plattform eines Samsung Galaxy Tab 10.1 entwickelt worden, da das System zusammen mit der Hardware-Plattform zur Verfügung gestellt wird.
\\
Für die Entwicklung der Applikation kommt die Java \gls{ide} Eclipse (Version Indigo) zur Anwendung. Die Datenpersistierung auf dem Tablet wird mithilfe der ORMLite Library (Version 4.39) in einer \gls{sqlite} Datenbank umgesetzt.
\\
Der Release 1 wurde an der Berner Rundfahrt 2012 getestet. Die dort gewonnenen Erkenntnisse wurden im Release 2 berücksichtigt, so dass nun ein System vorliegt, welches die Grundanforderungen erfüllt. Letzte, kleinere Anpassungen wird der Industriepartner im Rahmen der Systemübernahme vornehmen. Die neue \textit{RadioTour} Anwendung wird voraussichtlich bei der Tour de Suisse 2012 zum Einsatz kommen.
