\chapter{Ergebnisse und Schlussfolgerungen}
In diesem Kapitel werden die erreichten Ziele und das Endprodukt zusammengefasst und im Rahmen der Aufgabenstellung beurteilt. Des weiteren wird die Zukunft des Produktes und die weitere Entwicklung erörtert.

\section{Endprodukt}
Das Ziel war es, eine verbesserte und modernisierte Applikation für den \textit{RadioTour Speaker} zu erstellen, welche die bisherige Web Applikation ersetzt. Die Anforderungen bestanden aus den Features der bestehenden Lösung und wurden weitgehend erreicht. Zudem wurden die Wünsche und Anregungen des \textit{RadioTour Speakers} implementiert. Das Endprodukt beinhaltet die Android Applikation \textit{RadioTour} mit der passenden Hardware, einem Samsung Galaxy Tab 10.1. Diese Lösung ist im heutigen Stand, für die Begleitung von  Radrennen, einsatzfähig. Dies konnte durch den Einsatz an der Berner Rundfahrt bestätigt werden. Für die Entwicklung von weiteren Features und der Vorbereitung für die Verwendung an der Tour de Suisse wird die Applikation dem Industriepartner, der cnlab AG, übergeben.
\\

Gegen Ende des Projekts kamen noch weitere Anforderungen hinzu. 
So z.B. der Import der Fahrer direkt vom Server via \gls{json} oder das automatische Navigieren zur aktuellen Rennposition in der Marschtabelle. Da diese Anforderungen erst bei einem fortgeschrittenen Stadium des Projekts hinzukamen, konnten diese Features nicht mehr implementiert werden. Es handelt sich bei diesen Anforderungen nicht um kritische Bereiche, daher sind die Änderungen auch in Form eines Updates noch möglich.

\section{Ausblick}
Der nächste und letzte Feldtest der Applikation vor der Tour de Suisse erfolgt an den Radsporttage in Gippingen\footnote{Radsporttage Gippingen, \url{http://www.gippingen.ch/}, Aufgerufen am 30.05.2012.}. Da dieser Feldtest erst nach der Projektabgabe stattfindet, liegt er in der Obhut der cnlab AG. Das Feedback aus diesem Test wird zeigen, wie gut die Rückmeldungen aus der Berner Rundfahrt umgesetzt werden konnten.
\\

Bis zur Tour de Suisse gibt es also noch die Möglichkeit, Anpassungen vorzunehmen. Nach der Tour de Suisse, dem ersten Einsatz in einem Etappenrennen, ist eine weitere Auswertung und allfällige Optimierung sinnvoll. Mit dieser Arbeit wurde ein wichtiger Grundstein gelegt, mit dem weiter gearbeitet werden kann.