\chapter{Analyse}
Die Analyse untersucht die bestehende Web Applikation und fasst die Anforderungen zusammen. Weiter wird die Struktur der Applikation schematisch dargestellt. Es werden die verwendeten Technologien sowie einen kurzen Exkurs zu anderen Lösungen angesprochen.

\section{Requirements}
Die Anforderungen an die \textit{RadioTour} Applikation ergeben sich aus den Features der bisherigen Web Applikation und den Verbesserungsvorschlägen des \textit{RadioTour} Speakers. Im Anhang \ref{ref:usecases} sind sämtliche UseCases der bisherigen Applikation aufgeführt. Im folgenden Abschnitt werden die Funktionen, welche in diesem Projekt implementiert sind, aufgeführt.

\subsection{Funktionale Anforderungen}
Die funktionalen Anforderungen sind in drei Prioritätsstufen eingeteilt:

\textbf{zwingenden Anforderungen (must)}
\begin{itemize}
\item Fahrerliste, Etappen und Marschtabellen importieren
\item Fahrer ansehen, sortieren und bearbeiten
\item Gruppen bilden und Rückstand angeben
\item Gruppen auflösen
\item Events für Fahrer erfassen (Sturz, Arzt, Aufgabe)
\item Rennsituation an den Server übermitteln
\item Spezialklassemente und Wertungen erstellen und Fahrer zuweisen
\item Maillots erfassen und bearbeiten

\end{itemize}


\textbf{optionalen Anforderungen (can)}
\begin{itemize}
\item aktuelle Rennkilometer und Rennzeit anzeigen
\item Stoppuhr
\item aktuelle Position durch GPS bestimmen
\item 
\end{itemize}


\textbf{wünschenswerten Anforderungen (nice to have)}
\begin{itemize}
\item aktuelle Position in der Marschtabelle anzeigen.
\item \gls{splashscreen}
\end{itemize}


\section{Struktur der Applikation}
%ja, plural ist Status, http://de.wikipedia.org/wiki/Status
Die Applikation hat im Grunde zwei Status, einerseits werden vor dem Rennen die Fahrerliste und die Marschtabelle importiert andererseits wird die Rennsituation während dem Rennen erfasst und Änderungen festgehalten. Diese beiden Status können aber nicht absolut voneinander getrennt werden, da während dem Rennen Änderungen denkbar sind. So entsteht die Baumartige Struktur, wie sie in der Abbildung zu sehen ist.

\begin{figure}[h!]
\caption{Struktur der Applikation}
\centering
\includegraphics{05bericht/images/struktur.png}
\end{figure} 



\section{Technologien}
\subsection{Android}
Die native Programmiersprache für das Android Betriebssystem ist Java. Die Programmierung in Java bringt den Vorteil, dass auf die gesamte \gls{api} von Android zugegriffen werden kann. Weiter sind die Geräte genau dafür ausgelegt und die optimale Performance kann erreicht werden. Sämtliche Komponenten dieser Arbeit sind in Java geschrieben. Für die Persistierung der Daten auf dem Tablet wird eine \gls{sqlite} Datenbank verwendet.

\subsection{Externe Libraries}
Android beinhaltet bereits ein umfangreiches Framework zur Entwicklung. Einzig beim \gls{orm}, also beim Abbilden von Objektdaten in der Datenbank kommt eine externe Library zum Einsatz.
\\
\textit{ORMLite} \footnote{ORMLite, \url{http://ormlite.com/}, Aufgerufen am 23.05.2012} ist eine OpenSource Java Library, welche auch für Android eine optimale Lösung bietet. Die zu verwendenden Felder einer Klasse können mit Java Annotationen versehen werden, daraus versucht ORMLite dann die Datenbank zu beschreiben. In der RadioTour Anwendung konnten alle Felder abgebildet werden.


\subsection{Entwicklungsumgebung}
Die von Android empfohlene Entwicklungsumgebung ist Eclipse\footnote{Eclipse, \url{http://eclipse.org/}} mit einem Plugin zur Entwicklung von Android Applikationen. Auf der Entwicklerseite von Android steht dazu folgendes:
\begin{quote}
\grqq Android Development Tools (ADT) is a plugin for the Eclipse IDE that is designed to give you a powerful, integrated environment in which to build Android applications.\grqq
\footnote{Android Plugin für Eclipse, \url{http://developer.android.com/sdk/eclipse-adt.html}}
\end{quote}
Eclipse ist eine weit verbreitete \gls{ide} und wird aktiv weiter entwickelt. Mit dem Plugin zusammen bilden Sie eine solide Grundlage für dieses Projekt.
\\
Damit die Android Applikation direkt auf dem Computer getestet werden kann, stellt Google ein Emulator zur Verfügung. Der Emulator ist allerdings auch als solcher zu betrachten da die Bedienung nicht vergleichbar ist mit einem richtigen Tablet.

\subsection{Android Version}
Eine Anwendung wird für eine spezifische Android Version entwickelt und getestet, somit kann garantiert werden, dass das Verhalten der Anwendung  immer gleich ist. In dieser Arbeit ist dies die Version 3.1 mit dem Versionsnamen \textit{Honeycomb}.\footnote{\url{Android Version Honeycomb, http://de.wikipedia.org/wiki/Android_(Betriebssystem)\#Versionsverlauf}}.
\\
Die Entwicklung auf einer Version schliesst jedoch nicht aus, dass die Anwendung in neueren Versionen nicht mehr lauffähig ist. Auch \textit{RadioTour} kann für zukünftige Versionen weiterentwickelt und verwendet werden.

\section{Mitbewerberanalyse}
Die Art der Applikation ist sehr spezifisch und kann nicht direkt auf andere Sportereignisse angewendet werden. Deshalb beinhaltet die Analyse von Mitbewerbern nur die grossen europäischen Radrennen. Wie bei der Tour de Suisse ist auch in Frankreich an der \textit{Tour de France}\footnote{Tour de France, \url{http://www.letour.fr/}} ein RadioTour Speaker mit dabei. Darüber wie die Aufzeichnungen in Frankreich im genauen stattfinden kann aber nur spekuliert werden da die Informationen nicht öffentlich zugänglich sind.
\\
In Italien findet zum Zeitpunkt dieser Arbeit der \textit{Giro d'Italia}\footnote{Giro d'Italia, \url{http://www.gazzetta.it/Speciali/Giroditalia/2012/}} statt. Bei diesem Radrennen ist es möglich aus den Informationen, welche auf der Webseite verfügbar sind, zu schliessen, dass ein ähnliches System verwendet wird. Während dem Rennen ist es möglich die aktuelle Rennsituation zu betrachten.

\begin{figure}[h!]
\caption{Rennsituation am Giro d'Italia}
\label{fig:giro}
\includegraphics[scale=0.7]{05bericht/images/giro.png}
\end{figure} 

In der Abbildung \ref{fig:giro} ist der Live Abschnitt der offiziellen Webseite zu sehen. Im oberen Teil wird der Standort in der aktuelle Etappe eingeblendet. Weiter unten ist die Situation an der Spitze abgebildet. Die Fahrer sind nach Rückstand gruppiert.
\\
Da jedoch nicht zu erkennen ist, wie die Informationen im Feld erfasst werden, muss die Mitbewerberanalyse an dieser Stelle abgeschlossen werden.