% makeglossaries bericht.glo

\newacronym{rup}{RUP}{Rational Unified Prozess}
\newacronym{ide}{IDE}{Integrated Development Environment}

\newacronym{api}{API}{Application Programming Interface}
\newacronym{orm}{ORM}{Object-relational mapping}

\newglossaryentry{sqlite}{
	name=SQLite,
	description={SQLite ist eine Cross Plattform Datenbankengine, welche ohne Konfiguration auskommt. Es handelt sich dabei um eine Datenbank in einer Datei},
	first={SQLite}
}

\newglossaryentry{json}{
	name={JSON},
	description={JavaScript Object Notation (JSON), ist eine Notation zur Darstellung von Objekten in Textform},
	first={JavaScript Object Notation (JSON)}
}

\newglossaryentry{enum}{
	name={Enum},
	description={Ein Enum ist ein Datentyp mit fest bestimmten Konstanten. Es kann immer nur ein Wert ausgewählt sein.}
}

\newglossaryentry{splashscreen}{
	name={Splashscreen},
	description={Eine Anzeige, die oftmals beim Start einer Applikation die Wartezeit bis zur vollständigen Initialisierung überbrückt.}
}
\newglossaryentry{csv}{
	name={CSV},
	description={Ein Dateiformat, bei welchem die Datensätze über ein Trennzeichen voneinander getrennt sind.},
	first={Comma Separated Values (CSV)}
}
\newglossaryentry{http}{
	name={HTTP},
	description={Das Hypertext Transfer Protocol (HTTP) ist ein weit verbreitetes Protokoll, um Daten und Inhalte im Web zu übertragen. In dieser Arbeit wird es in der Kommunikation mit dem Server verwendet.},
	first={Hypertext Transfer Protocol (HTTP)}
}
\newglossaryentry{junit}{
	name={JUnit Test},
	description={Java bietet die Möglichkeit integrierte Softwaretests automatisiert durchzuführen. Dies erleichtert die Arbeit enorm und unterstützt ein Entwicklungsteam, eine möglichst hohe Testabdeckung zu erarbeiten},
	first={Java Unit Test (JUnit)}
}

\newglossaryentry{chronofunk}{
	name={Chronofunk},
	description={Die Motorradfahrer, welche im Rennfeld verteilt mitfahren und die Positionen der Ausreissergruppen per Funk an den RadioTour Speaker übermitteln}
}