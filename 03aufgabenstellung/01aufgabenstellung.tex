\chapter*{Aufgabenstellung}

\begin{tabular}{ll}
Studiengang: & Informatik (I)\\
Semester: & FS 2012 (20.02.2012-16.09.2012)\\
Institut: & ITA: Internet-Technologien und Anwendungen\\
Gruppe: & Florian Bentele, Daniel Stucki\\
Verantwortlicher: & Dr. Prof. Peter Heinzmann, pheinzma@hsr.ch\\
Industriepartner: & cnlab AG, Lukas Frey, lukas.frey@hsr.ch

\end{tabular}

\section*{Ausgangslage}
Bei Radrennen erfasst der so genannte \textit{RadioTour Speaker} Informationen zur Rennsituation, welche ihm von Motorradfahrern per Funk geliefert werden. Gegenwärtig legt der \textit{RadioTour Speaker} mit Hilfe einer TabletPC Web-Anwendung per Click auf die erhaltenen Fahrernummern die Zusammensetzung der Gruppen fest. Er tippt auch die Zeitabstände zwischen den Gruppen ein. Die \textit{RadioTour} Anwendung  visualisiert die Fahrergruppen und liefert Detailinformationen zu den Fahrern (z.B. Namen, Team, virtueller Rang). Veränderungen in den Fahrergruppen können per Drag-and-Drop nachgeführt werden. Die mit der \textit{RadioTour} Anwendung erfassten Rennsituationen werden per Mobilfunknetz zu einem Webserver gesendet, wo sie weiteren Anwendungen, z.B. Live Webinformationen zur Verfügung stehen.

\section*{Ziel}
Die aus dem Jahre 2006 stammende Notebook Web Applikation soll nun dahingehend überarbeitet und erweitert werden, dass man sie auf Android-Tablets oder iPads betreiben kann.

\section*{Teilaufgaben}
\begin{itemize}
\item Analyse der existierenden \textit{RadioTour} und TourLive-Anwendungen
\begin{itemize}
\item Tour de Suisse Dokumente
\item Alte \textit{RadioTour} Anwendung\\
\url{http://gps.cnlab.ch/tablet/}\\
user: ba\_tourlive\\
password: access4tl
\item TourLive-System (GPS Positions- und Bilderfassungssysteme, Webanwendung)
\item Kommunikation \textit{RadioTour} – TourLive-Webanwendung
\item Vergleich mit Systemen anderer Radrennen 
\end{itemize}

\item Festlegung der Funktionalität der neuen Tablet-Anwendung (Requirements Engineering)
\begin{itemize}
\item Studium der Geschäftsprozesse (Renninformationen, Rennverlauf)
\item Austesten von Teilfunktionen (Android-Tablet Programmierung, GPS-Positionserfassung, Usability Experimente)
\item Erweiterte Funktionen (z.B. Erfassung der Streckenkilometer, Integration von Marschtabellen)
\item Auswahl der Hardware-Plattform
\item Spezifikation
\end{itemize}

\item Design
\begin{itemize}
\item Benutzerschnittstelle
\item Kommunikation mit TourLive Aufnahmesystemen (Weitergabe der Daten an Datenserver)
\item \textit{RadioTour} Anwendung
\end{itemize}

\item Realisierung
\begin{itemize}
\item Prototypen
\item Anpassungen
\item Friendly User Test Version (Beta-Release)
\item Feldtest an einem Radrennen
\item Übergabe an cnlab
\end{itemize}

\item Dokumentation
\begin{itemize}
\item Gemäss Anforderungen Industriepartner (das System soll vom Industriepartner betrieben und erweitert werden können)
\item Bericht gemäss HSR / Heinzmann Richtlinien
\end{itemize}
\end{itemize}

\section*{Abgrenzung}
Das Produkt wird spezifisch auf ein Gerät ausgerichtet und nicht plattformübergreifend entwickelt. Die Verbindung zum Server wird in der Arbeit definiert, jedoch werden keine serverseitigen Entwicklungen erarbeitet.
\\
Die Mehrsprachigkeit wird nach Android Standards implementiert \footnote{Android Internationalisierung, \url{http://developer.android.com/guide/topics/resources/localization.html}}. Eine Übersetzung ist jedoch nicht Teil der Arbeit.
\\
Diese Aufgabenstellung wird genehmigt vom Betreuer
\\
\\

\begin{tabular}{p{3cm}p{4cm}}
\hline
Ort / Datum & P. Heinzmann
\end{tabular}

\section*{Erklärung}
Ich erkläre hiermit, dass ich die vorliegende Arbeit selber und ohne fremde Hilfe durchgeführt habe, ausser derjenigen, welche explizit in der Aufgabenstellung erwähnt ist oder mit dem Betreuer schriftlich vereinbart wurde, dass ich sämtliche verwendeten Quellen erwähnt und gemäss gängigen wissenschaftlichen Zitierregeln korrekt angegeben habe.
\\
Rapperswil, 29. Mai 2012
\\
\\

\begin{tabular}{p{5cm}p{5cm}}
\hline

Florian Bentele & Daniel Stucki
\end{tabular}
